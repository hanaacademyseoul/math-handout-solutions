%! TeX program = xelatex
\documentclass{article}
\usepackage{amsmath}
\usepackage{amssymb}
\usepackage{fontspec}
\usepackage[normalem]{ulem}
\usepackage[margin=0.75in]{geometry}

\setmainfont{Noto Serif CJK KR}
\title{1. 집합의 뜻과 표현 step C 풀이}
\author{Project Eclipse}
\date{}

\begin{document}
\maketitle
\paragraph{1번}
$n \in A_q$이려면 $n \le \frac{p^2}{q} < n + 1$의 정수해가 존재해야 한다.

\subparagraph{$q = 3$인 경우}
다음과 같은 부등식

\[
3 \le \frac{p^2}{3} < 4
\]

이 정수해 $p = 3$을 가지므로 $3 \in A_3$이다.

\subparagraph{$q = 4$인 경우}
부등식을 세워 보면

\[
4 \le \frac{p^2}{4} < 5
\]

이 부등식은 정수해가 없으므로 $3 \notin A_4$이다.

\subparagraph{$q = 5$인 경우}
부등식을 다시 세워 보면

\[
5 \le \frac{p^2}{5} < 6
\]

이 부등식은 정수해 $p = 5$를 가지므로 $3 \in A_k$이다. 따라서 정답은 \underline{4}번이다.

\paragraph{2번}
이 문제는 간단하다. 정의를 그대로 따라가면 $x_i = 1$인 경우 $i \in X$가 성립한다. \newline 따라서 $x_1 = x_3 = x_4 = 1$이고, $x_2 = 0$이므로 $X$는 \underline{$\{1, 3, 4\}$}이다.

\paragraph{3번}
각각의 보기에 대해 참/거짓을 판별해 보자.

\subparagraph{보기 1번}
모든 집합 $A$에 대해 $\varnothing \subset A$가 성립하므로 참이다.

\subparagraph{보기 2번}
보기 1번과 같은 이유로 참이다.

\subparagraph{보기 3번}
$\{ \varnothing \} \subset 2^A$는 $\varnothing \in 2^A$와 같은 말이므로 참이다.

\subparagraph{보기 4번}
$A = \{0, 1, 2\}$일 때, $2^A = \{\varnothing, \{0\}, \{1\}, \{2\}, \{0, 1\}, \{1, 2\}, \{2, 0\}, \{0, 1, 2\}\}$이므로 $A \not\subset 2^A$, 거짓이다.

\subparagraph{보기 5번}
$n(2^A)$는 $A$의 부분집합의 개수이므로 $2^{n(A)}$와 같고, 따라서 참이다. \newline

정답은 \underline{4}번이다.

\paragraph{4번}
규칙대로 계산해 보면 $A * A = \{0, 1, 2\}$이고, $A \odot A = \{-1, 0, 1\}$ 이므로 ㄱ, ㄷ이 맞는다. 따라서 정답은 \underline{3}.

\paragraph{5번}
$a \equiv b$일 때, $a$와 $b$를 5로 나눈 나머지는 같다. 사실 이 문제는 이 성질을 이용하여 $0 \le x < 8$인 정수에 대해 모두 확인해도 풀 수 있다. 그렇게 푸는 것이 오히려 빠를 수도 있지만, 별로 얻을 수 있는 것이 별로 없고 \sout{내가 이 프로젝트에서 잘릴 수도 있기 때문에} 좀 더 제대로 된 풀이로도 풀어보자. \newline

$k$가 음이 아닌 정수이고, $r$이 $0 \le r < 5$인 정수일 때, $5k + r$을 세제곱하면 다음과 같을 것이다.

\[
125k^3 + 75rk^2 + 15r^{2}k + r^3 = 5(25k^3 + 15rk^2 + 3r^{2}k) + r^3
\]

이 식을 통해 $(5k + r)^3$을 5로 나눈 나머지는 $r^3$을 5로 나눈 나머지와 같다는 것을 알 수 있다. 따라서 $B = \{3\}$이고, 부분집합의 개수는 $2^1 = \underline{2}$임을 알 수 있다.

\end{document}
